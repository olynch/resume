\documentclass{article} 
\usepackage[margin=0.5in]{geometry}
\usepackage{graphicx}
\usepackage{Alegreya}
\usepackage{tikz,tikz-cd}
\usepackage{hyperref}
\hypersetup{
  colorlinks=true,
  linkcolor=blue,
  urlcolor=blue,
}

\renewcommand{\section}[1]{\vspace{1.5em}{\huge \textsc{#1}}\vspace{1em}}
\renewcommand{\subsection}[1]{{\large \textbf{#1}}\vspace{0.5em}}
\newcommand{\descdate}[2]{#1 \hfill #2\vspace{0.5em}}

\newenvironment{row}{\vspace{0.5em}\begin{minipage}[t]{\textwidth}}{\end{minipage}\vspace{0.5em}}
\newenvironment{column2}{%
  \begin{minipage}[t]{0.5\textwidth}%
    \begin{minipage}[t]{0.9\textwidth}}{%
    \end{minipage}%
  \end{minipage}}
\newenvironment{column3}{%
  \begin{minipage}[t]{0.333\textwidth}%
    \begin{minipage}[t]{0.9\textwidth}}{%
    \end{minipage}%
  \end{minipage}}
  
\setlength{\parindent}{0pt}
\pagenumbering{gobble}

\begin{document}

{\huge Owen Lynch} \hfill \texttt{\href{mailto:owen\_lynch1@brown.edu}{owen\_lynch1@brown.edu}}

\hrule

\section{Education}

\begin{row}
  \begin{column2}
    \subsection{Brown University}

    \descdate{Pursuing Bachelor of Science in Math}{May 2020}

    Selected Courses: Topology (Graduate), Complex Analysis, Reading Program in Category Theory, Topics in Abstract Algebra, Functions of Several Variables, Partial Differential Equations, Probability, Algorithms
  \end{column2}
  \begin{column2}
    \subsection{Commonwealth School}

    \descdate{Diploma}{June 2016}

    Selected Courses: Abstract Algebra, Theoretical Calculus, Mathematical Logic, Introduction to AI, Introduction to Compilers
  \end{column2}
\end{row}
\begin{row}
  \begin{column2}
    \subsection{Budapest Semesters in Mathematics}

    \descdate{Semester Abroad}{Spring 2019}

    Expected Courses: Combinatorics, Graph Theory, Lie Groups, Analytic Number Theory
  \end{column2}
\end{row}
  
\section{Writings and Talks}

\begin{row}
  \begin{column2}
    \subsection{The Inherent Instability of Disordered Systems}

    Paper written at NECSI that introduces a way of measuring the change in a complexity profile, and uses this framework along with Ashby's law to talk about how complex systems change over time. \href{https://arxiv.org/abs/1812.00450}{arxiv.org/abs/1812.00450}
  \end{column2}
  \begin{column2}
    \subsection{Meditations on Tensors}

    Blog post explaining tensors from several different angles, intended to give intuition that might not be found elsewhere. \href{https://owenlynch.org/posts/2019-01-02-meditations-on-tensors/}{owenlynch.org/posts/2019-01-02-meditations-on-tensors/}
  \end{column2}
\end{row}
\begin{row}
  \begin{column2}
    \subsection{FizzBuzz In Haskell}

    A talk introducing Haskell through three different presentations of FizzBuzz. \href{https://owenlynch.org/notes/FizzBuzz\%20In\%20Haskell.pdf}{owenlynch.org/notes/FizzBuzz In Haskell.pdf}
  \end{column2}
\end{row}

\section{Experience}

\begin{row}
  \begin{column2}
    \subsection{Tesla Motors}

    \descdate{Firmware Intern}{Summer 2018}

    Developed software in Haskell to facilitate development of internal car network. Designed new algorithm for tracking redundancy and failure-resilience in communication components.
  \end{column2}
  \begin{column2}
    \subsection{New England Complex Systems Institute}

    \descdate{Student Researcher}{2016-2017}

    Researched and implemented novel techniques in image processing. Wrote large-scale data scraping infrastructure. Helped to migrate large amounts of data. Studied novel information theoretic methods for modeling changes in complex systems. Worked in Python, R, and C.%
  \end{column2}
\end{row}

\begin{row}
  \begin{column2}
    \subsection{Brown CS Department}

    \descdate{Algorithms TA}{Fall 2018}
  \end{column2}
  \begin{column2}
    \subsection{Brown Math Department}

    \descdate{Math Resource Center Tutor}{2017}
  \end{column2}
\end{row}

% \section{Technical Proficiencies}

% \newcommand{\proficiency}[2]{%
%   \begin{minipage}[t]{1.6cm}
%     \centering

%     #1
    
%     \vspace{1em}

%     {\large \textbf{#2}}
%   \end{minipage}\hspace{1em}}

% \newcommand{\proficiencyext}[2]{\proficiency{\def\svgwidth{\textwidth} \input{#1}}{#2}}

% \proficiencyext{haskell.pdf_tex}{Haskell}
% \proficiencyext{c.pdf_tex}{C/C++}
% \proficiencyext{python.pdf_tex}{Python}
% \proficiencyext{rust.pdf_tex}{Rust}
% \proficiencyext{nixos.pdf_tex}{NixOS}
% \proficiencyext{linux.pdf_tex}{Linux}
% \proficiencyext{git.pdf_tex}{Git}
% \proficiencyext{mathematica.pdf_tex}{Wolfram}
% \proficiency{\includegraphics[width=\textwidth]{matlab.png}}{Matlab}

\begin{row}
  \begin{column3}
    \section{Proficiencies}
    Haskell, C/C++, Python, MATLAB, Mathematica, Linux, Git, \LaTeX{} (obviously). I am a very capable programmer, and I also have familiarity with many technologies not listed here.
  \end{column3}
  \begin{column3}
    \section{Interests}

    Category Theory and Applied Category Theory, Differential Geometry, Type Theory and Homotopy Type Theory, Algebraic Topology, Intuitionist Logic, Formal Verification, Algorithms, Learning New Things
  \end{column3}
  \begin{column3}
    \section{Miscelaneous}

    I play violin in the Brown University Orchestra as well as chamber groups and pit orchestras, and I swing dance.
  \end{column3}
\end{row}
\end{document}








% Local Variables:
% TeX-engine: luatex
% End:
